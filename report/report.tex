\documentclass[a4peper, 12pt, titlepage, finall]{extreport}

%различные пакеты

\usepackage[T1, T2A]{fontenc}
\usepackage[russian]{babel}
\usepackage[justification=centering]{caption}
\usepackage[backend=bibtex]{biblatex}
\usepackage{csquotes}
\usepackage{tikz}
\usepackage{geometry}
\usepackage{indentfirst}
\usepackage{fontspec}
\usepackage{graphicx}
\usepackage{array}
\usepackage{pgfplots}
\usepackage{fancyvrb}
\usepackage{listings}
\graphicspath{{./images/}}

\usetikzlibrary{positioning, arrows}

\makeatletter
\newcount\dirtree@lvl
\newcount\dirtree@plvl
\newcount\dirtree@clvl
\def\dirtree@growth{%
  \ifnum\tikznumberofcurrentchild=1\relax
  \global\advance\dirtree@plvl by 1
  \expandafter\xdef\csname dirtree@p@\the\dirtree@plvl\endcsname{\the\dirtree@lvl}
  \fi
  \global\advance\dirtree@lvl by 1\relax
  \dirtree@clvl=\dirtree@lvl
  \advance\dirtree@clvl by -\csname dirtree@p@\the\dirtree@plvl\endcsname
  \pgf@xa=1cm\relax
  \pgf@ya=-1cm\relax
  \pgf@ya=\dirtree@clvl\pgf@ya
  \pgftransformshift{\pgfqpoint{\the\pgf@xa}{\the\pgf@ya}}%
  \ifnum\tikznumberofcurrentchild=\tikznumberofchildren
  \global\advance\dirtree@plvl by -1
  \fi
}

\tikzset{
  dirtree/.style={
    growth function=\dirtree@growth,
    every node/.style={anchor=north},
    every child node/.style={anchor=west},
    edge from parent path={(\tikzparentnode\tikzparentanchor) |- (\tikzchildnode\tikzchildanchor)}
  }
}
\makeatother


\geometry{a4paper, left = 15mm, top = 10mm, bottom = 15mm, right = 15mm}
\setmainfont{Spectral Light}%{Times New Roman}
\setcounter{secnumdepth}{0}
%\setcounter{tocdepth}{3}
\nocite{*}

\begin{document}
\begin{titlepage}
    \begin{center}
	{\small \sc Московский государственный университет \\имени М.~В.~Ломоносова\\
	Факультет вычислительной математики и кибернетики\\}
	\vfill
	~\\
	{\Large Копмьютерный практикум по учебному курсу}\\
	~\\
	{\large \bf \sc <<РАСПРЕДЕЛЕННЫЕ СИСТЕМЫ>>}\\ 
	~\\
	~\\
	{\large \bf \sc ЗАДАНИЯ}\\
	~\\
	{\large \bf Круговой алгоритм выбора координатора}
	~\\
	~\\
	{\large \bf Доработка MPI программы, реализованной в рамках курса "Суперкомпьютеры и параллельная обработка данных" с целью улучшить надежность}
	~\\
	~\\
	{\large \bf \sc ОТЧЕТ}\\
	~\\
	{\large \bf о выполненном задании}\\
	~\\
	{\large студента 421 учебной группы факультета ВМК МГУ}\\
	~\\
	{\large Никифорова Н.И.}\\
	~\\
	\vfill
	{\small Москва\\2021}
    \end{center}
\end{titlepage}
    \section{Постановка задачи}
        Необходимо было выполнить следующие два пункта:
        \begin{itemize}
            \item  В транспьютерной матрице размером 5*5, в каждом узле которой находится один процесс, необходимо переслать очень длинное сообщение 
                (длиной L байт) из узла с координатами (0,0) в узел с координатами (4,4). Реализовать программу, моделирующую выполнение такой пересылки 
                на транспьютерной матрице с иcпользованием блокирующих операций MPI. Получить временную оценку работы алгоритма, 
                если время старта равно 100, время передачи байта равно 1 (Ts=100,Tb=1). Процессорные операции, включая чтение из памяти и запись в память, считаются бесконечно быстрыми.
            \item Доработать MPI-программу, реализованную в рамках курса “Суперкомпьютеры и параллельная обработка данных”. Добавить контрольные точки 
                для продолжения работы программы в случае сбоя. Реализовать один из 3-х сценариев работы после сбоя: a) продолжить работу 
                программы только на “исправных” процессах; б) вместо процессов, вышедших из строя, создать новые MPI-процессы, которые необходимо использовать 
                для продолжения расчетов; в) при запуске программы на счет сразу запустить некоторое дополнительное количество MPI-процессов, которые использовать в случае сбоя.
        \end{itemize}
        \section{Структура проекта}
            Код проекта доступен в открытом репозиторие по ссылке \textbf{https://github.com/Rav263/ski\_2020}, код предыдущего задания находится по ссылке 
            \textbf{https://github.com/Rav263/SKI\_adi2d}. Проект состоит из двух частей, для каждого пункта задачи соответственно.
                \begin{flushleft}
                \begin{tikzpicture}[dirtree]
                    \node {ski\_2020}
                        child { node {part\_1}
                            child { node {main.cpp} }
                        }
                        child { node {part\_2}
                            child { node {Adi2D\_MPI\_error\_correction.c}}
                            child { node {Adi2D\_MPI.c} }
                        }
                        child { node {report}
                            child { node {report.tex} }
                            child { node {report.pdf} }
                        };
                \end{tikzpicture}
            \end{flushleft}
        \section{Сборка проекта}
            \subsection{Первый пункт (блокирующий алгоритм передачи сообщения)}
\begin{lstlisting}
cd ./part_1
mpic++ main.cpp
mpirun -n 25 --oversubscribe ./a.out
\end{lstlisting}
            \subsection{Второй пункт (Улучшение существующего алгоритма Adi2D)}
                Для сборки второй части программы требуется установленный пакет ulfm (Можно найти на сайте \textbf{fault-tolerance.org}).
    Сборка ulfm (флаги для \textbf{configure} взяты из системного mpi:\\
\begin{lstlisting}
git clone https://bitbucket.org/icldistcomp/ulfm2.git
cd ulfm2
git submodule update --init --recursive
./autogen.pl
./configure ’--prefix=/usr’ ’--sysconfdir=/etc/openmpi’\
            ’--enable-mpi-fortran=all’\
            ’--libdir=/usr/lib/openmpi’\
            ’--enable-builtin-atomics’ ’--enable-mpi-cxx’\
            ’--with-valgrind’ ’--enable-memchecker’\
            ’--enable-pretty-print-stacktrace’\
            ’--without-slurm’ ’--with-hwloc=/usr’\
            ’--with-libltdl=/usr’ ’FC=/usr/bin/gfortran’
make all
sudo make install
\end{lstlisting}
~\\
            Сборка алгоритма: 
~\\
\begin{lstlisting}
cd ./part_2
mpicc Adi2D\_MPI\_error\_correction.c
mpirun -n 6 --oversubscribe ./a.out
\end{lstlisting}
    \section{Тестовый стенд}
        \begin{itemize}
            \item Процессор $-$ AMD Ryzen 2700X 8/16 (ядер/потоков) 4.2 Ghz на одно ядро, 4.0 Ghz на все ядра.
            \item Память $-$ 32Gb DDDR4 3200Mhz (Пропускная способность 44GB/s).
        \end{itemize}
    \section{Алгоритм блокирующей пересылки сообщения на транспьютерной матрице}
        На рисунке~\ref{fig:dep} изображена транспьютерная матрица, желтым цветом обозначены стартовая и конечная вершина, красным и зелёным соответственно первый и второй пути передачи сообщения.
        В начале программы происходит инициализация процессов в клетках транспьютерной матрице. Затем первый процесс начинает рассылку сообщения. 
        Время старта равно 100, время передачи одного байта равно 1. \(Ts = 100, Tb = 1\).
        Введём некоторые обозначения:
        \begin{itemize}
            \item Длина сообщения $-$ \(L\) (байт);
            \item Количество кусков, на которое делится сообщение $-$ \(K\);
            \item Количество путей $-$ \(P = 2\) так как пути не должны пересекаться;
            \item Размер одного сообщения $-$ \(N = \frac{L}{P * K}\) (байт);
        \end{itemize}

        Теперь рассчитаем время передачи одного сообщения. Из процесса \((0, 0\) до процесса \((4, 4)\), происходит \(8\) передач. 
        Соответственно на передачу одного куска сообщения потребуется \(8 * (Ts + Tb * \frac{L}{P * K}) = 8 * (Ts + Tt * N)\).
        (На инициализацию канала передачи)
        А время передачи остальных кусков сообщения \((K - 1) * (Ts + Tb * N)\).
        Тогда общее время передачи будет равно:\\
        \begin{center}
            $T_{all} = 8 * (Ts + Tb * \frac{L}{P * K}) + (K - 1) * (Ts + Tb * \frac{L}{P * K})$\\
            $T_{all} = 8 * (Ts + Tb * N) + (K - 1) * (Ts + Tb * N)$
        \end{center}

        Но так как в условии говорится об \textit{очень длинном сообщении} можно пренебречь временем старта передачи сообщения, длиной маршрута и временем разгона конвейера.
        Таким образом у нас остаётся только \(Tb\):
        \begin{center}
            $T_{all} = L * Tb / P$
        \end{center}

        \begin{figure}[htbp]
            \centering
            \includegraphics[width=0.7\textwidth]{matrix.png}
            \caption{Транспьютерная матрица}\label{fig:dep}
        \end{figure}
    \section{Улучшение алгоритма \textbf{Adi2D}}
        Необходимо было улучшить алгоритм {\tt Adi2D} с целью улучшения надёжности.
        Был выбран вариант с созданием дополнительных процессов при старте программы, которые будут использоваться в случае сбоя.
        Также каждые 5 итераций каждый процесс сохраняет свою часть матрицы в файл $-$ контрольные точки.

        Описание внесённых изменений:
        \begin{enumerate}
            \item Переменная {\tt \bf additional\_procs} задаёт количество дополнительных процессов, который создаются на старте.
            \item Функция {\tt \bf verbose\_errhandler} отвечает за обработку ошибок, возникших во время работы программы.
                В начале выполнения данной функции вызывается {\tt MPIX\_Comm\_revoke}, чтобы прервать все текущие операции
                общения и все оставшиеся процессы попали в данную функцию. В конце данной функции происходит перераспределение рангов, очищается
                выделенная память и происходит выделение новой $-$ соответствующей по размеру рангу процесса. Затем выставляется флаг ошибки, который позволяет
                пропустить, после выхода из функции обработки ошибок, все дальнейшие операции вычисления до новой итерации. В начале итерации, если стоит флаг
                ошибки происходит загрузка данных с последней контрольной точки, и флаг ошибки сбрасывается.
            \item Функция {\tt \bf save\_matrix()} сохраняет матрицу в файл {\tt matrix\_{\it rank}}, где {\tt \it rank} $-$ ранг процесса, выполняющего запись.
            \item Функция {\tt \bf load\_matrix()} загружает матрицу из файла {\tt matrix\_{\it rank}}, где {\tt \it rank} $-$ ранг процесса, выполняющего загрузку.
            \item Переменная {\tt \bf last\_save\_it} хранит номер последний итерации на которой была сделана контрольная точка.
        \end{enumerate}

        \subsection{Экспериментальное исследование}
            Было проведено небольшое экспериментальное исследование, целью которого было выяснить процентное падение производительности относительно не модифицированной версии программы.
            А также выяснить время, которое занимает восстановление программы. Исходя из прошлого исследования был выбран размер матрицы \(8192\),
            как самый оптимальный для подсчёта на 8 потоках.

            Как видно на рисунке~\ref{graph:1} с увеличением количества потоков и увеличивается разрыв во времени выполнения старой и улучшенной версий, это связано с количеством записей в файлы,
            так как чем больше число потоков, тем больше времени занимает запись. В процентном соотношении разрыв составил \(13\% ... 90\%\).
            
            Рассмотрим теперь время затрачиваемое на восстановления после возникновения ошибки. Как видно из рисунка~\ref{graph:2} чем больше процессов используется, тем больше времени занимает восстановление, при одинаковом количестве ошибок.

            \begin{figure}[!htbp]
                \centering
            \begin{tikzpicture}
            \begin{axis}[
                width=0.8\textwidth,
                height=0.6\textwidth,
                ylabel={Время, секунды},
                xlabel={Количество потоков},
                y label style={at={(-0.05, 0.5)}},
                ymin=0, ymax=45,
                xmin=0, xmax=10,
                xtick={0, 1, 2, 3, 4, 5, 6, 7, 8},
                ytick={5, 10, 15, 20, 25, 30, 35, 40},
                legend pos=south east,
                ymajorgrids=true,
                grid style=dashed,
                cycle list name=color list] 
                \addplot+[mark=square] coordinates{(1, 34.591)(2, 25.401)(4, 21.079)(6, 20.50)(8, 19.549)};
                \addplot+[mark=square] coordinates{(1, 30.502)(2, 18.450)(4, 12.648)(6, 11.051)(8, 10.271)};
                \legend{New, Old}
     
            \end{axis}
            \end{tikzpicture}
                \captionsetup{justification=centering}
                \caption{Зависимость времени исполнения от количества потоков}
                \label{graph:1}
            \end{figure}
            
            \begin{figure}[!htbp]
                \centering
            \begin{tikzpicture}
            \begin{axis}[
                width=0.8\textwidth,
                height=0.6\textwidth,
                ylabel={Время, секунды},
                xlabel={Количество ошибок},
                y label style={at={(-0.05, 0.5)}},
                ymin=0, ymax=60,
                xmin=0, xmax=10,
                xtick={0, 1, 2, 3, 4, 5, 6, 7, 8},
                ytick={5, 10, 15, 20, 25, 30, 35, 40, 45, 50, 55},
                legend pos=south east,
                ymajorgrids=true,
                grid style=dashed,
                cycle list name=color list] 
                \addplot+[mark=square] coordinates{(1, 3.321)(2, 9.964)(4, 32.942)(6, 50.241)};
                \addplot+[mark=square] coordinates{(1, 10.317)(2, 19.182)(4, 39.09)(6, 49.95)};
                %\addplot+[mark=square] coordinates{(1, 11.16)(2, 16.512)};
                \legend{1 процесс, 4 процесса}
     
            \end{axis}
            \end{tikzpicture}
                \captionsetup{justification=centering}
                \caption{Зависимость времени восстановления от количества ошибок}
                \label{graph:1}
            \end{figure}
\end{document}
